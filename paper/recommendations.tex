\section{Situtational Recommendations}
As we saw in Section \ref{sec:recommender}, the hybrid recommendation
system is the most effective in making better suggestions.  {\sys}
will adopt a hybrid recommendation system with weighted majority
voting, where we will take into account several situational factors,
including time, location, mood-based content filtering, traditional
social network collaborative filtering, and cross-media collaborative
filtering.

\subsection{Time and Location}
A unique feature of our recommendation algorithm is the situational
factors, specifically the time and location that a viewer watches a
program.  Viewers are definitely influenced by the presence of their
friends, and they may only watch something if their friends are with
them.  For example, let's say that Mary only likes to watch Community
with her friends on Thursday nights, but wouldn’t normally watch it by
herself.  We can track that she only watches it on Thursday nights
with the same group of people and weight Community less in the
recommendation algorithm if Mary is watching alone, but highly
recommend it if Mary is with those same friends.  We can also look at
the time and location factors individually.  If at night, you are more
prone to watching late night cartoons, we can weight those more
highly.  In addition, if you tend to watch sports only at bars, we
won’t recommend sports to you while you’re watching at home alone.
Adding this temporal and locational component to recommendations
allows us to suggest programs more tailored to your individual
schedule and activities.

\subsection{Mood/Specialized Genres}
The Pandora style tagging model we've adopted will also help factor
into our recommendations.  This will contribute to the content
collaborative filtering.

\subsection{Traditional Social Network Filtering}
Following Pinterest and Spotify’s model of automatically opting in the
user’s facebook profile, {\sys} will also grab viewer's facebook
information in order to access your friends.  Since the emphasis of
{\sys} is watching with friends, we must be able to have access to
your friends in order to recommend the appropriate content.  Another
benefit of using Facebook is that we can then leverage Facebook’s
social graph to find the viewer’s closest friends, and weight the
collaborative filtering aspect of our algorithm appropriately.

We could also link to viewer's other available social networks. After
running some natural language processing algorithms on user’s blogs,
tweets, and status updates, we can gain other valuable information on
the user’s likes and dislikes.

\subsection{Cross-media collaborative filtering}
In addition to the prior collaborative filtering methods, we can
introduce cross-media collaborative filtering into our recommendation
system.  The goal here is to develop profiles of users based on their
tastes, using similarity in profiles to do tv show suggestions.

We look at three different kinds of cross-media data analytics:

\begin{itemize}
\item \textbf{Browsing history}.  The browser history of a user is a
  valuable signal in classifying the user.  For example, a person who
  frequently reads espn.com would probably enjoy a very different tv
  show than someone who spends a lot of time reading fashion websites.
  These are signals into interests and consequently television show
  preferences.
\item \textbf{Purchase history}.  Similarly, tracking a user's
  purchase history provides insight into preferences and tastes.  A
  store can determine if a patron is pregnant just by her shopping
  history~\cite{target}, and this kind of information is a strong
  signal into what content that person might want to watch on tv.
\item \textbf{News/Sharing history.}  Reading material provides
  insight about high/low end consumer preferences.  Users who read the
  New York Times and read the New Yorker might have a tendency to
  watch HBO dramas, while users who read gossip websites prefer
  reality television.
\end{itemize}
